\documentclass{standalone}
%\usepackage{fullpage}
\usepackage{tikz}
\usepackage{graphicx}

\usetikzlibrary{positioning}
%\usetikzlibrary{shapes.multipart}

\tikzstyle{rootClass}=[rectangle,black,draw, fill=orange!30, text width=5 cm]
\tikzstyle{essentialClass}=[rectangle,black,draw, fill=green!30, text width=5 cm]
\tikzstyle{exampleClass}=[rectangle,black,draw, fill=red!30, text width=5 cm]
\tikzstyle{narrowBox}=[text width=5 cm]
\tikzstyle{wideBox}=[text width=13 cm]

\tikzstyle{inheritedFrom}=[->,black, line width=2]
\tikzstyle{instantiatedIn}=[-,black, line width=2] 
%Note that A instantiatedIn B relationship reduces flexibility as the 
%children of A cannot be known to B now.

\newcommand{\nodeTextWidthRatio}{0.95}
\newcommand{\spaceafter}{\vspace{0mm}}
\newcommand{\method}[1]{\spaceafter\textbf{\texttt{\detokenize{#1}}}\\}
\newcommand{\desc}[1]{\ \ \ \texttt{\detokenize{#1}}\\}
\newcommand{\className}[1]{\large \textbf{\detokenize{#1}}}
\newcommand{\data}[1]{\spaceafter\textbf{\texttt{\detokenize{#1}}}\\}
\newcommand{\code}[1]{\verb+ #1 +\\}

\def\showExampleClasses{1}
\def\showEssentialClasses{1}
%Note that you can't have showExampleClasses=1 with showEssentialClasses=0
%as the former nodes refer to the latter and this will cause error.

\begin{document}

\begin{tikzpicture}


%NOTE: you can't refer to exampleClass and essentialClass nodes
% in definition of roots. Order of definition matters for that and
%other type of nodes come later. 
\node[rootClass, wideBox] (World) {
	\begin{tabular}{p{\nodeTextWidthRatio\textwidth}} 

\className{World}
\\  \hline
\data{objects: list[Object]}

 \\ \hline 
\method{constructor(map: Map) }

\method{evolve(delta-t)}

\desc{calls evolve() of all objects which have no owner, takes care of offspring objects, and finally kills objects which ask for it}

\method{run()}

\desc{manages delta-t (how small it should be), manages intersections using self.intersect(), and calls self.evolve()}

\method{intersect() -> intersection-result: list[InIn]}

\desc{instantiates InIn for each pair of objects and calls InIn.intersect() to evaluate the intersection}
\end{tabular} 


};
\node[rootClass, wideBox] (Object) [below=of World] {
	\begin{tabular}{p{\nodeTextWidthRatio\textwidth}} 
\className{Object}
\\ \hline
\data{objectID}

\desc{every object needs an ID to be tracable in map and in state}

\data{shape: Shape}

\desc{which could be empty if the object is an owner}

\data{position: Position}

\desc{position and orientation of an anchor point of the object}
\\ \hline
\method{evolve(delta-t, intersection-result: InIn) -> list[Object]: offspring-objects}

\desc{changes the state (position, internal attributes, etc) of the object}

\desc{trvial evolution: when the object never changes state}

\desc{offsprings are the possibly non-physical objects required to accomplish something.}

\method{visualize()}

\desc{returns the information required for visualization}

\method{bounding-box() -> Box}

\desc{returns a box which contains the whole object. used to optimize intersection evaluation}

\method{get-required-delta-t()}

\desc{calculates the delta-t it requires to operate}

\method{time-to-die() -> bool}

\desc{tells the World if it wants to be eliminated. This might be where Agent Smith cheated the matrix!}
\end{tabular} 

};

\node[rootClass, wideBox ] (Map) [right=of World] {
	\begin{tabular}{p{\nodeTextWidthRatio\textwidth}} 
\className{Map}
\\  \hline
\data{next_available_id}
 \\ \hline 

 \method{parse_map(self, filename: str) -> dict[ObjectId, Object]}

 \method{get_next_id(self) -> ObjectId}

 \method{instantiate_object(obj_json, new_id: ObjectId, name: string, owner: Object) -> Object}

\method{get_shape(obj_json) -> Shape}

\method{get_position(obj_json) -> Shape}

\method{get_objects(self, obj_map: dict[ObjectId, Object], parsed, owner: Object) -> dict[ObjectId, Object]}
\end{tabular} 


};

\node[rootClass ] (Shape) [right=of Map] {
	\begin{tabular}{p{\nodeTextWidthRatio\textwidth}} 
\className{Shape}
\\ \hline
\\ \hline
\method{bounding_box()}
\desc{returns a x-y plane bounding box. Can be done using a generalized algorithm, no implemented only in the parent class.}

\method{dump_info()}
\end{tabular} 


};

\node[rootClass ] (Position) [above=of Shape] {
	\input{nodes/rootClasses/Position}
};

\node[rootClass,wideBox ] (InIn) [left=of World] {
	\begin{tabular}{p{\nodeTextWidthRatio\textwidth}} 
\className{InIn}

\desc{IntersectionInstance}
\\ \hline

\\ \hline
\method{intersect() -> dict}

\desc{evaluates the intersection. An algorithm with mathematical calculations.}

\end{tabular}
};

% \node[rootClass ] (WorldState) [right=of World] {
% 	\input{nodes/rootClasses/WorldState}
% };

%\draw [instantiatedIn] (Object) -- (World);

\if\showEssentialClasses1
	%NOTE: you can't refer to exampleClass nodes in definition of 
	%essentials or roots. Order of definition matters for that and
	%exampleClasses come later. 
	% \node[essentialClass ] (XMLmap) [below=of Map] {
	% 	\input{nodes/essentialClasses/XMLmap}
	% };

	% \node[essentialClass ] (DiscretizedShape) [below=of Shape] {
	% 	\input{nodes/essentialClasses/DiscretizedShape}
	% };

	% \node[essentialClass ] (DiscretizedShape2D) [below=of DiscretizedShape] {
	% 	\input{nodes/essentialClasses/DiscretizedShape2D}
	% };

	\node[essentialClass, wideBox ] (RigidPhysicalObject) [right=of Object] {
		\input{nodes/essentialClasses/RigidPhysicalObject}
	};

	\node[essentialClass ] (Robot) [left=of Object] {
		\begin{tabular}{p{\nodeTextWidthRatio\textwidth}} 
\className{Robot}
\\ \hline
\\ \hline
\desc{evolve() is the heart of Robot actions.}
\end{tabular} 


	};

	\node[essentialClass ] (Sensor) [left=of Robot, below=of InIn] {
		\begin{tabular}{p{\nodeTextWidthRatio\textwidth}} 
\className{Sensor}

\desc{inherently an Object, so needs shape, position, evolve(), etc.}
\\ \hline
\\ \hline
\method{sense()} 
\end{tabular} 


	};

	\draw [inheritedFrom] (Robot) -- (Object);
	\draw [inheritedFrom] (Sensor) -- (Object);
	\draw [inheritedFrom] (RigidPhysicalObject) -- (Object);
	% \draw [inheritedFrom] (DiscretizedShape) -- (Shape);
	% \draw [inheritedFrom] (DiscretizedShape2D) -- (DiscretizedShape);
	% \draw [inheritedFrom] (XMLmap) -- (Map);
\fi

\if\showExampleClasses1
	% \node[exampleClass ] (Disk) [below=of Shape, right=of DiscretizedShape] {
	% 	\input{nodes/exampleClasses/Disk}
	% };

	\node[exampleClass ] (Cylinder) [below=of Shape] {
		\begin{tabular}{p{\nodeTextWidthRatio\textwidth}} 
\className{Cylinder}
\\ \hline
\\ \hline
\end{tabular} 


	};

	\node[exampleClass ] (Cube) [right=of Cylinder] {
		\begin{tabular}{p{\nodeTextWidthRatio\textwidth}} 
\className{Cube}
\\ \hline
\\ \hline
\end{tabular} 


	};
	% \node[exampleClass ] (Wheel) [below=of Object] {
	% 	\input{nodes/exampleClasses/Wheel}
	% };

	% \node[exampleClass ] (DiningTable) [right=of Wheel] {
	% 	\input{nodes/exampleClasses/DiningTable}
	% };

	\node[exampleClass ] (VaccumCleaner) [below=of Robot] {
		\begin{tabular}{p{\nodeTextWidthRatio\textwidth}} 
\className{VaccumCleanerV0}

\desc{in version 0, a cylinder with no wheels and stuff.}
\\ \hline
\\ \hline
\method{evolve()}
\desc{overriden}

\method{te_required_delta_t()}
\desc{overriden}
\end{tabular} 
	};

	\node[exampleClass ] (Box) [below=of Object] {
		\begin{tabular}{p{\nodeTextWidthRatio\textwidth}} 
\className{Box}
\\ \hline
\\ \hline
\end{tabular} 


	};

	\node[exampleClass ] (BumperSensor) [below=of Sensor] {
		\begin{tabular}{p{\nodeTextWidthRatio\textwidth}} 
\className{BumperSensor}

\desc{detects at a hit upon intersection, boolean result}
\\ \hline
\\ \hline
\end{tabular} 


	};

	% \node[exampleClass ] (LidarSensor) [left=of BumperSensor] {
	% 	\input{nodes/exampleClasses/LidarSensor}
	% };

	\node[exampleClass, wideBox ] (RigidPointBall) [right=of RigidPhysicalObject] {
		\begin{tabular}{p{\nodeTextWidthRatio\textwidth}} 
\className{RigidPointBall}

\desc{A ball with relevant calculations for reflection upon bump, but not rotating}

\desc{v0: 2D cylinder instead of sphere.}
\\ \hline
\\ \hline
\end{tabular} 
    
    
	};

	% \draw [inheritedFrom] (Disk) -- (Shape);
	\draw [inheritedFrom] (Cylinder) -- (Shape);
	\draw [inheritedFrom] (Cube) -- (Shape);
	% \draw [inheritedFrom] (DiningTable) -- (Object);
	\draw [inheritedFrom] (Box) -- (Object);
	\draw [inheritedFrom] (VaccumCleaner) -- (Robot);
	\draw [inheritedFrom] (BumperSensor) -- (Sensor);
	% \draw [inheritedFrom] (LidarSensor) -- (Sensor);
	\draw [inheritedFrom] (RigidPointBall) -- (RigidPhysicalObject);

\fi

%%%%% the matrix annotation part :D

%\node[inner sep=0pt, right of= InIn, xshift=-6cm, yshift=1cm] (neo)
%    {\includegraphics[width=.25\textwidth]{neo.jpg}};

%\node[inner sep=0pt, right of= neo, xshift=-4.5cm] (architect)
%    {\includegraphics[width=.25\textwidth]{architect.jpg}};

%\node [rectangle,black,draw, below of=architect, yshift=-2cm, text width = 10cm] 
%    {\textbf{Neo} : Why am I here?\\
%	\textbf{The Architect} : Your life is the sum of a remainder of an unbalanced equation inherent to the programming of the matrix. You are the eventuality of an anomaly, which despite my sincerest efforts I have been unable to eliminate from what is otherwise a harmony of mathematical precision.};

% \node[inner sep=0pt, right of= Map, xshift=5cm] (smith)
%     {\includegraphics[width=.4\textwidth]{smith.jpg}};

% \node [rectangle,black,draw, below of=smith, yshift=-1.5cm, text width = 8cm] 
%     {\textbf{Smith to Neo}: Because Of You, I'm No Longer An Agent Of This System. Because Of You, I've Changed. I'm Unplugged. A New Man, So To Speak.};


\end{tikzpicture}

\end{document}
